\documentclass[a4paper, 12pt]{book}

\usepackage{fancyhdr}

\newcommand{\ttitle}{Numerične metode za linearne sisteme upravljanja - zapiski s predavanj prof. Plestenjaka}
\newcommand{\tauthor}{Tomaž Poljanšek}
\newcommand{\tdate}{študijsko leto 2023/24}

\usepackage{color}
\usepackage{soul}
\usepackage[numbers]{natbib}

\usepackage{physics}

\usepackage[parfill]{parskip}
\usepackage[hyphens]{url}

\usepackage[usestackEOL]{stackengine}[2013-10-15] % formatting Pascal
\usepackage[dvipsnames]{xcolor}

\usepackage{cancel}
\usepackage[export]{adjustbox}

% Related to math
\usepackage{amsmath,amssymb,amsfonts,amsthm}
\usepackage{mathtools}
\usepackage{youngtab}
\usepackage{tikz}
\usepackage{yhmath}

% encoding and language
\usepackage{lmodern}
\usepackage[slovene, english]{babel}
\usepackage[utf8]{inputenc}
\usepackage[T1]{fontenc}

% multiline comments
\usepackage{comment}
\usepackage{verbatim}

% random text - for texting
\usepackage{lipsum}
\usepackage{blindtext}

\usepackage{hyperref}

\usepackage{listings}
\usepackage{verbatim}
\usepackage{fancyvrb}

% images
\usepackage{graphicx}
\graphicspath{ {../images/} }

% no blank page
\usepackage{atbegshi}
\renewcommand{\cleardoublepage}{\clearpage}

% theorems
\theoremstyle{definition}
\newtheorem{counter}{Counter}[section]
\newtheorem{defn}[counter]{Definicija}
\newtheorem{lemma}[counter]{Lema}
\newtheorem{conseq}[counter]{Posledica}
\newtheorem{claim}[counter]{Trditev}
\newtheorem{theorem}[counter]{Izrek}
\newtheorem{pro}[counter]{Dokaz}
%%
\theoremstyle{remark}
\newtheorem*{ex}{Primer}
\newtheorem*{exmp}{Zgled}
\newtheorem*{rem}{Opomba}

% QED
\renewcommand\qedsymbol{$\blacksquare$}

\hypersetup{pdftitle={\ttitle}}

\addtolength{\marginparwidth}{-20pt}
\addtolength{\oddsidemargin}{40pt}
\addtolength{\evensidemargin}{-40pt}

\renewcommand{\baselinestretch}{1.3}
\setlength{\headheight}{15pt}
\renewcommand{\chaptermark}[1]
{\markboth{\MakeUppercase{\thechapter.\ #1}}{}} \renewcommand{\sectionmark}[1]
{\markright{\MakeUppercase{\thesection.\ #1}}} \renewcommand{\headrulewidth}{0.5pt} \renewcommand{\footrulewidth}{0pt}

% header
\fancyhf{}
\fancyhead[LE,RO]{\sl \thepage} 
\fancyhead[RE]{\sc \tauthor}
\fancyhead[LO]{\sc Numerične metode za linearne sisteme upravljanja}


\newcommand{\autfont}{\Large}
\newcommand{\titfont}{\LARGE\bf}
\newcommand{\clearemptydoublepage}{\newpage{\pagestyle{empty}\cleardoublepage}}
\setcounter{tocdepth}{1}

\usepackage{bbold}

\newcommand{\N}{\mathbb{N}}
\newcommand{\Z}{\mathbb{Z}}
\newcommand{\Q}{\mathbb{Q}}
\newcommand{\R}{\mathbb{R}}
\newcommand{\C}{\mathbb{C}}
\newcommand{\F}{\mathbb{F}}
\newcommand{\Po}{\mathbb{P}}
\newcommand{\One}{\mathbb{1}}

\newcommand{\ch}{\operatorname{char}}
\newcommand*\Eval[3]{\left.#1\right\rvert_{#2}^{#3}}
\newcommand*\circled[1]{\tikz[baseline=(char.base)]{%
            \node[shape=circle,fill=white!20,draw,inner sep=2pt] (char) {#1};}}

\makeatletter
\newcommand{\longeq}[1]{\mathrel{\mathpalette\longeq@{#1}}}
\newcommand{\longeq@}[2]{%
  \begingroup
  \sbox\z@{$\m@th#1=$}%
  \ifdim#2<\wd\z@
    \resizebox{#2}{\height}{\box\z@}%
  \else
    \ifdim#2<3\wd\z@
      \hbox to #2{$\m@th#1=\hss=\hss=\hss=$}%
    \else
      \hbox to #2{$\m@th#1=\cleaders\hbox to 0.2\wd\z@{\hss$#1=$\hss}\hfil=$}%
    \fi
  \fi
  \endgroup
}
\makeatother

\DeclarePairedDelimiter\ceil{\lceil}{\rceil}
\DeclarePairedDelimiter\floor{\lfloor}{\rfloor}

\usepackage{float}
\usepackage{multirow}
\usepackage{icomma}
\usepackage{tabularx}
\usepackage{hhline}

\usepackage{enumitem}
\usepackage{ulem}
\newcommand{\msout}[1]{\text{\sout{\ensuremath{#1}}}} % cross text in math mode

\title{\ttitle}
\author{\tauthor}
\date{\tdate}

\newcommand\mymaketitle{
  \begin{titlepage}
    \begin{center}
        \vspace*{4cm}
        \Huge
        \textbf{\ttitle}
                        
        \vspace{1.5cm}
        \huge
        \tauthor
            
        \vspace{3cm}
        \Large
        \tdate
    \end{center}
  \end{titlepage}
}




\begin{document}

\selectlanguage{slovene}
%\setcounter{page}{1}
\renewcommand{\thepage}{}
\newcommand{\sn}[1]{"`#1"'}

\mymaketitle

\clearpage

\frontmatter

% kazalo
%\pagestyle{empty}
%\def\thepage{}
%\tableofcontents{}

%%
%\def\x{\hspace{3ex}}    %BETWEEN TWO 1-DIGIT NUMBERS
%\def\y{\hspace{2.45ex}}  %BETWEEN 1 AND 2 DIGIT NUMBERS
%\def\z{\hspace{1.9ex}}    %BETWEEN TWO 2-DIGIT NUMBERS
%\stackMath

%\clearpage
%\phantomsection

%\section*{Seznam uporabljenih kratic}

%\noindent\begin{tabular}{p{0.1\textwidth}|p{.8\textwidth}}
%  {\bf kratica} & pomen \\ \hline
%  {\bf SVM}   & support vector machine (metoda podpornih vektorjev) \\
%\end{tabular}

%\clearpage
%\phantomsection
%\addcontentsline{toc}{chapter}{Povzetek}
%\chapter*{Povzetek}

%Predloga.


\mainmatter
\setcounter{page}{1}
\pagestyle{fancy}

\pagenumbering{arabic}



% 1. predavanje 20.2.

\chapter{Klasična teorija}


\section{Sistemi upravljanja}

Imamo dinamični sistem, sestavljen iz več komponent. \\
% skica
Stanje sistema opisujejo notranje spremenljivke, nanj vplivamo (upravljamo, vodimo) z vhodom $u(t)$,
opazujemo pa lahko izhod $y(t)$. \\
Vhodno-izhodna oblika. \\
$t:$ čas, \\
$u(t) \in \R^m$, \\
$y(t) \in \R^r$, \\
$x(t) \in \R^n$, \\
$n >> m, r$. \\
Upravljanje običajno poteka preko krmilnika (regulatorja). \\
% skica
Sisteme ločimo na
\begin{enumerate}[label=\alph*)]
  \item odprtozančne in
  \item zaprtozančne.
\end{enumerate}
Pri odprtozančnih sistemih krnilnik ni povezan z izhodom (stanjem) sistema. \\
Npr.
\begin{enumerate}[label=-]
  \item ročna klimatska naprava,
  \item stari parni stroji,
  \item glasbene skrinjice,
  \item svetilnik.
\end{enumerate}
Pri zaprtozančnih sistemih imamo še povratno zvezo s stanjem ali izhodom sistema. \\
% skica
Zgledi:
\begin{itemize}[label=-]
  \item avtomatska klimatska naprava,
  \item tempomat,
  \item avtopilot,
  \item kotliček za izplakovanje,
  \item Wattov regulator parnega stroja.
\end{itemize}
% skica
Manj pretoka $\implies$ počasneje, več pretoka $\implies$ hitreje.


\section{Lastnosti sistemov}

Splošni dinamični sistem lahko predstavimo s pomočno preslikave iz vhodnih funkcij v izhodne funkcije. \\
Vpeljimo naslednje oznake: \\
$T$: časovni prostor, urejena podmnožica $\R$, \\
$U$: vhodni prostor, množica vseh možnih stanj vhoda, $\subset \R^m$, \\
$\Omega \subset \{u: T \to U\}$: prostor vseh možnih vhodnih funkcij, \\
$X$: prostor stanj, množica vseh možnih stanj sistema, $\subset \R^n$. \\
Če ima sistem izhod, imamo še \\
$Y$: izhodni prostor, množica vseh možnih stanj izhoda, $\subset \R^r$, \\
$\Gamma \subset \{y: T \to Y\}$: prostor vseh izhodnih funkcij. \\
$\Omega$ mora biti neprazen in za $t_1 < t_2 < t_3$ iz $T$ in poljubni $u_1, u_2 \in \Omega$ mora obstajati $u_3 \in \Omega$:
\begin{equation*}
  u_3(t) = \begin{cases*}
    u_1(t), & za $t_1 \leq t \leq t_2$ \\
    u_2(t), & za $t_2 \leq t \leq t_3$
  \end{cases*}
\end{equation*}
Naš sistem opisuje preslikavo stanja \\
$\phi: T \times T \times X \times \Omega \to X$, kjer je \\
$\phi(t_1, t_0, x_0, u)$ stanje sistema $x(t_1)$ v času $t_1 \in T$, ki nastane iz začetnega stanja $x_0 \in X$
v času $t_0 \in T$ pod vplivom vhodne funkcije $u \in \Omega$. \\
$\phi$ mora biti dobro definirana za $t_1 \geq t_0$, ne pa tudi za $t_1 < t_0$. \\
Za $\phi$ mora veljati:
\begin{enumerate}[label=\alph*)]
  \item lastnost identitete: $\phi(t_0, t_0, x_0, u) = x_0 \;$ $\forall t_0 \in T, \forall x_0 \in x, \forall u \in \Omega$,
  \item lastnost podgrupe: $t_0 \leq t_1 \leq t_2 \in T: \;$ $\phi(t_2, t_0, x_0, u)
    = \phi(t_2, t_1, \phi(t_1, t_0, x_0, u), u)$.
\end{enumerate}
% skica
Če ima sistem izhod, obstaja še preslikava \\
$\psi: T \times X \times U \to Y$, da je \\
$y(t) = \psi(t, x(t), u(t))$ stanje izhoda v času $t$. \\
$\to$ izhod je odvisen samo od trenutnega stanja sistema in vhoda v času $t$ in časa $t$.
\begin{defn}
  Sistem je vzorčen, če je za poljuben $t_1 \in T$ velja: \\
  Če za $u_1, u_2 \in \Omega$ velja $u_1(t) = u_2(t)$ za $\forall t \leq t_1$, potem je \\
  $\phi(t_1, t_0, x_0, u_1) = \phi(t_1, t_0, x_0, u_2)$ za $\forall t_0 \leq t_1 \in T, \forall x_0 \in X$.
\end{defn}
Vzročnost pomeni, da je stanje sistema odvisno samo od prejšnjih ali sedanjih vrednosti vhoda.
\begin{defn}
    Naj bosta $\Omega$ in $X$ vektorska prostora.
    Sistem je linearen, če je za $\forall t_0 \leq t_1 \in T$ funkcija $\phi(t_1, t_0, ., .)$ linearna.
\end{defn}
$\phi(t_1, t_0, \alpha_1 x_1 + \alpha_2 x_2, \alpha_1 u_1 + \alpha_2 u_2)
= \alpha_1 \phi(t_1, t_0, x_1, u_1) + \alpha_2 \phi(t_1, t_0, x_2, u_2)$ \\
$\alpha_1 \begin{bmatrix} x_1 \\ u_1 \end{bmatrix} + \alpha_2 \begin{bmatrix} x_2 \\ u_2 \end{bmatrix}
= \begin{bmatrix} \alpha_1 x_1 + \alpha_2 x_2 \\ \alpha_1 x_1 + \alpha_2 u_2 \end{bmatrix}$ \\
za $\forall x_1, x_2 \in X, \forall u_1, u_2 \in \Omega$, skalarja $\alpha_1, \alpha_2$. \\
Če ima sistem izhod, mora biti $y$ vektorski prostor in $\psi(t, ., .)$ linearna za $\forall t \in T$. \\
Če je sistem linearen, iz
\begin{equation*}
    \begin{bmatrix} x_0 \\ u \end{bmatrix} = 1 \cdot \begin{bmatrix} x_0 \\ 0 \end{bmatrix}
    + 1 \cdot \begin{bmatrix} 0 \\ u \end{bmatrix}
\end{equation*}
dobimo
\begin{equation*}
    \phi(t_1, t_0, x_0, u) = \phi(t_1, t_0, x_0, 0) + \phi(t_1, t_0, 0, u);
\end{equation*}
$\phi(t_1, t_0, x_0, 0)$: odziv na ničelni vhod (zero input response), \\
$\phi(t_1, t_0, 0, u)$: odziv z ničelnim stanjem (zero state response). \\
% skica
\begin{lemma}
    Če je sistem linearen, je vzorčnost ekvivalentna pravilu začetnega mirovanje (p.z.m.).
\end{lemma}
Če za $u \in \Omega$ velja $u(t) = 0 \; \forall t \leq t_1$, potem je $\phi(t_1, t_0, 0, u) = 0 \; \forall t_0 \leq t_1$.
\begin{pro} \text{} \\
    $(\Rightarrow):$ \\
        Denimo, da sistem ne zadošča p.z.m. \\
        Torej $\exists \tilde{u} \in \Omega$, $\tilde{u} = 0$ za $t \leq t_1$
        in za nek $t_0 \leq t_1$ je $\phi(t_1, t_0 0, \tilde{u}) \neq 0$. \\
        Potem za poljubne $u$ in $x_0$ velja \\
        $\phi(t_1, t_0, x_0, u + \tilde{u}) \neq \phi(t_1, t_0, x_0, u)$,
        toda $u + \tilde{u}$ in $u$ se ujemata na $t \leq t_1$. \\
        $\implies$ sistem ni vzročen. \\
    $(\Leftarrow)$: \\
        Če sistem ni vzročen, $\exists u_1$ in $u_2$, ki se ujemata na $t \leq t_1$ in \\
        $\phi(t_1, t_0, x_0, u_1) \neq \phi(t_1, t_0, x_0, u_2)$. \\
        $\implies$ če vzamemo $\tilde{u} = u_1 - u_2$, je $\tilde{u} = 0$ na $t \leq t_1$ in $\phi(t_1, t_0, 0, \tilde{u}) \neq 0$.
\end{pro}
\begin{lemma}
    Če je sistem linearen, je $\phi(t_1, t_0, 0, 0) = 0$.
\end{lemma}
\begin{pro}
    $\phi(t_1, t_0, \alpha x_0, \alpha u) = \alpha \phi(t_1, t_0, x_0, u)$, vstavimo $\alpha = 0$.
\end{pro}
% skica
Za $\sigma \in T$ definiramo operator premika: $u \to u^{\sigma}$, kjer je $u^{\sigma}(t) = u(t - \sigma)$. \\
Velja naj, da je $T$ aditivna grupa in $\Omega$ zaprta za operator premika za \\
$\forall \sigma \in T$.
\begin{defn}
    Pravimo, da je sistem časovno nespremenljiv (time invariant), če za
    $\forall t_0 \leq t_1 \in T, \forall x_0 \in X, \forall u \in \Omega, \forall \sigma \in T$:
    \begin{equation*}
        x(t_1) := \phi(t_1, t_0, x_0, u) = \phi(t_1 + \sigma, t_0 + \sigma, x_0, u^{\sigma}) =: x^{\sigma}(t_1).
    \end{equation*}
\end{defn}
Če ima izhod, mora biti $\phi$ neodvisna od $t$.
\begin{exmp} \text{} \\
    \begin{center}
        \begin{tabular}{c | c c c}
             & vzorčen & linearen & časovno invarianten \\
            \hline
            $x(t) = u^2(t-1)$ & ja & ne & ja \\
            $x(t) = u(-t)$ & ne & ja & ne \\
            $x(t) = 3^{-t} u(t-1)$ & ja & ja & ne
        \end{tabular}
    \end{center}
\end{exmp}
Mi se bomo ukvarjali z vzorčnimi LTI (linearen + časovno invarianten) sistemi.
Ukvarjali se bomo z naslednjimi oblikami vzorčnih LTI sistemov.
\begin{enumerate}[label=\alph*]
    \item Zvezni sistemi: $T = \R$.
    \begin{enumerate}[label=a\arabic*)]
        \item Klasična vhodno-izhodna oblika. \\
            $y^{(n)}(t) + k_1 y^{(n-1)}(t) + \dots + k_n y(t) = \beta_0 u^{(m)}(t) + \dots + \beta_m u(t)$ \\
            začetni pogoji pri $t_0$.
        \item Predstavitev v prostoru stanj. \\
            $x^{.}(t) = A x(t) + B u(t); \quad x(t_0) = x_0, t \geq t_0$ \\
            $x^{.}(t) \in \R^n$: vektor stanja sistema, \\
            $u(t) \in \R^m$: vektor izhoda, \\
            $y(t) \in \R^r$: izhod, \\
            $m, r << n$. \\
            % skica
            $A: n \times n$ matrika stanje, \\
            $B: n \times m$: vhodna matrika, \\
            $C: r \times n$: izhodna matrika, \\
            $D: r \times m$: matrika diskretnega prehoda.
    \end{enumerate}
    \item Diskretni sistemi: $T = \{\delta t \cdot k, k \in \Z\}$; \\
        $\delta t$: interval vzorčenja. \\
        $u_k = u(k \cdot \delta t)$: iz diferencialnih enačb dobimo diferenčne enačbe
        \begin{enumerate}[label=b\arabic*)]
            \item $y_{j+n} + k_1 y_{j+n-1} + \dots + k_n y_j = \beta_0 u_{j+m} + \dots +  \beta_m u_j$,
                $j = 0, 1 \dots$ \\
                + začetne vrednosti.
            \item $x_{k+1} = A x_k + B u_k$ \\
                $y_k = C x_k + D u_k$.
        \end{enumerate}
\end{enumerate}
\begin{exmp} \text{} \\
    $x^{'}(t) = a x(t) + u(t)$, \\
    $x(t_0) = x_0, a \in \R$ \\
    $\implies x(t) = e^{a(t-t_0)} + \int_{t_0}^t e^{a(t-s)} u(s) ds = \phi(t, t_0, x_0, u)$. \\
    Preverimo lahko
    \begin{enumerate}[label=\alph*)]
        \item lastnost identitete: $\phi(t_0, t_0, x_0, u) = x_0 + 0 = x_0$,
        \item lastnost polgrupe:
            \begin{align*}
                &\phi(t_2, t_1, \phi(t_1, t_0, x_0, u), u) \\
                &= e^{a(t_2-t_1)} \cdot e^{a(t_1-t_0)} x_0
                + \int_{t_0}^{t_1} e^{a(t_1-s)} u(s) ds + \int_{t_1}^{t_2} e^{a(t_2-s)} u(s) ds \\
                &= \phi(t_2, t_0, x_0, u).
            \end{align*}
    \end{enumerate}
    Podobno preverimo linearnost:
    \begin{align*}
        &\phi(t_1, t_0, \alpha_1 x_1 + \alpha_2 x_2, \alpha_1 u_1 + \alpha_2 u_2) \\
        &= e^{a(t_1-t_0)}(\alpha_1 x_1 + \alpha_2 x_2) + \int_{t_0}^{t_1} e^{a(t_1-s)} (\alpha_1 u_1(s) + \alpha_2 u_2(s)) ds \\
        &= \alpha_1 \phi(t_1, t_0, x_1, u_1) + \alpha_2 \phi(t_1, t_0, x_2, u_2).
    \end{align*}
    Vzročnost = p.z.m. (pogoj začetnega mirovanja) \\
    $\phi(t_0, t_0, 0, 0) = 0$. \\
    Časovna nespremenljivost
    \begin{align*}
        &\phi(t_1 + \sigma, t_2 + \sigma, x_0, u^{\sigma}) \\
        &= e^{a(t_1+\sigma-(t_0+\sigma))} x_0 + \int_{t_0+\sigma}^{t_1+\sigma} e^{a(t_1+\sigma-s)} u(s-\sigma) ds.
    \end{align*}
    Substitucija $\tilde{s} = s - u$.
\end{exmp}



%\clearpage
%\phantomsection

%\addcontentsline{toc}{chapter}{Literatura}
%\bibliography{../bibtex/literatura}
%\bibliographystyle{plainnat}


%\clearpage
%\phantomsection

%\chapter*{Dodatki}
%\addcontentsline{toc}{chapter}{Dodatki}
%D.




\end{document}
